\documentclass{article}
\usepackage{graphicx, wrapfig, subcaption, setspace,booktabs,verbatim,hyperref, textcomp, siunitx}
\usepackage{float,placeins}
\usepackage{multirow}
\usepackage{epstopdf}
%\usepackage{multicol}
\usepackage[margin=0.7in]{geometry}

%%%%% Set all your parameters and paths in INIT_DMQC.M (not here) %%%%%%%%%%%%%%%%%%%%%%%
%\newcommand{\WMOnum}{6903550} %% FILL WITH WMONUM
\input{wmonumdef} % Defines \WMOnum.
% Full paths from root are likely necessary here:
\input{floatsourcedef} % Defines \floatsource.
%\newcommand{\floatsource}{/Users/a21627/Arkiv/data/matlab_owc/float_source/project_NorARGO/} 
\input{floatplotsdef} % Defines \floatplots.
%\newcommand{\floatplots}{/Users/a21627/Arkiv/data/matlab_owc/float_plots/project_NorARGO/} 
\input{downloaddirdef} % Defines \downloaddir.
%\newcommand{\downloaddir}{/Users/a21627/Downloads/ARGO/\WMOnum /} 
%%%%%%%%%%%%%%%%%%%%%%%%%%%%%%%%%%%%%%%%%%%%%%%%%%%%%%%%%%%%%%%%
 
\begin{document}

\title{Delayed Mode Quality Control of Argo float~\WMOnum}%\hspace{0.5mm}}
\author{\input{dmqc_operator_name}\\ %Jan Even {\O}ie Nilsen\\   %% Name of DMQC operator
ORCID: \input{dmqc_operator_orcid}\\ %0000-0003-2516-6106\\[1cm]
\input{dmqc_operator_institution}\\ %Institute of Marine Research (IMR)\\
\input{dmqc_operator_address} } %Strandgaten 196, Bergen, Norway} %% Name of Organisation and address
\date{\today} %% Date of DMQC if you produce report on same day as DMQC.
%\date{yyyy.mm.dd} %% Insert actual date of DMQC if you produce report later.
\maketitle

\vspace{-1cm}
\begin{center}
  \textbf{DMQC summary}
\end{center} 
% THIS FILE IS NECESSARY FOR YOUR DMQCreport_float.tex TO COMPILE.
% THIS IS WHERE YOU WRITE YOUR DISCUSSION/SUMMARY (WE HAVE DECIDED THAT THESE TWO CAN BE THE SAME, FOR SIMPLICITY).
% BELOW ARE SOME EXAMPLE PHRASES THAT YOU CAN COMMENT, UNCOMMENT, EDIT, OR REPLACE AS YOU NEED AND WISH FOR THIS PARTICULAR FLOAT.
%
Float number \WMOnum\ was deployed on \input{depdate}
% centrally in the Norwegian Basin (NB) and has only moved slightly southwestwards, but is still in the deep parts of the basin.
% Hydrography is relatively stable below 1000~m in the profiles.

% Visual verification of in total 232 RTQC flags in 7 profiles was done, of which two had large groups of deviating salinity readings. 
% The DMQC tests resulted in 2 further flags, all minor individual spikes.
% No RTQC flags where present, nor was any new flags necessary in DMQC. 
% The initial comparison between Argo float data and reference data, shows that temperature and salinity data are within normal values, somewhat on the warm and fresh side.
% The comparison with satellite altimeter data shows the same trends, but the float derived DHA has very little of the variability seen in altimetry data.
% The comparison with satellite altimeter data suggested no notable discrepancies.
% The sea surface pressure data are not displaying values below 0~dbar and there are no indications of negative pressure drift.

% The OWC analysis showed no noteworthy indications of drift or offset. Hence, no correction is done.
% The OWC analysis showed a strong negative salinity offset, and float salinity has been corrected with a $\Delta S \approx +0.01$.
% The offset is also evident in long term variability seen in the reference data (Figure~\protect\ref{fig:trendcheck}) and more recent CTD profiles near deployment (Figure~\protect\ref{fig:supplementary}).

\input{monitoring} % DO NOT TOUCH!

%%% Local Variables: 
%%% mode: plain-tex
%%% TeX-master: "DMQCreport_float"
%%% End: 

%\input{summary}

%\begin{flushleft}% text alignment to left
% In this section, write a brief description of all decisions made in DMQC
% analysis. You may include information about sea surface pressure
% corrections with applied QC flags and errors (if applicable), cell thermal
% mass corrections (if applicable), a decision made on salinity data
% including QC flags and corrections applied to salinity data (if needed).
% \vspace{0.5cm}

% For Example: "The sea surface pressure in Apex float was adjusted in
% d-mode. For cycles 1-155, the QC=1 and error 2.4 dbar was assigned to
% pressure data. Cell thermal mass correction was applied.  For cycles
% 1-155, the salty offset was detected. Correction of -0.0125 offset was
% applied, QC=1, error=0.005."

\tableofcontents

\newpage
\section{Introduction}
 
%Briefly write any necessary information about the float parameters,
%location, any steps performed before the DMQC analysis, version and type
%of the reference data, DMQC software and if there any CTD data from
%deployment was used as a reference data and any additional information
%about the deployment issues.     
%\vspace{0.5cm}

This report concerns the delayed mode quality control performed for
Argo-float number~\WMOnum. For more information about this float use, for
instance, the following link:
\begin{center}
\href{url}{http://www.ifremer.fr/argoMonitoring/float/\WMOnum}.  
\end{center}
Before the analysis, real-time QC flags were visually inspected and
modified if necessary.  In addition, a few stricter tests necessary before
the salinity calibration were applied (and flags modified if necessary).
% 
Every single profile has been visually inspected, and the validity of all
flags previously assigned flags considered and changed if deemed
necessary, and new flags assigned.
%
Then, the satellite altimeter comparison plot between the sea surface
height and dynamic height anomaly, constructed for this float by Ifremer,
was analysed.
%
Part of this analysis are also time series plots of pressure, temperature,
and salinity.
%and plots of temperature, salinity and density plotted against the nearby
%historical CTD and Argo profiles.
                                %
%The additional plots supporting the DMQC analysis are included in the ...

% \textbf{Float~\WMOnum\ is the Apex float, where the pressure sensor is not
%   auto-corrected to zero while at the sea surface, hence the pressure data
%   was corrected during processing in delayed mode.  The procedures of
%   correction sea surface pressure are described in Argo Quality Control
%   Manual for CTD and Trajectory Data (Wong et al., 2020).}

% \textbf{The cell thermal mass corrections were applied. For SBE-41 CTDs
%   the estimated alpha was 0.0267 and tau was 18.6 s, for the estimated
%   alpha was 0.141 and tau was 6.68 s (Johnson et al., 2006).}

The salinity calibration has been performed using the configuration and objective
mapping parameters included in Section~\ref{sec:salcompare}.
The Argo float data were compared to nearby CTD and Argo profiles from the
following reference databases: \small{{\verbatiminput{refdir}}}
%
Reference data are distributed by Ifremer. A simple visual check on the
reference data is done prior to analysis (see Appendix~B).
%
The OWC toolbox version~3.0.0 % Version number needs to be manually updated here
(\href{url}{https://github.com/ArgoDMQC/matlab\_owc}) was run to estimate
a salinity offset and a salinity drift (Cabanes et al., 2016).

Note that the concepts ``cycle number'' and ``profile number'' are
not the same, as there might be missing cycles, while only existing
profiles are input to OWC. ``Cycle number'' refers to the actual cycle
number parameter and file naming of R/D-files.

Only results for ascending profiles are included in this report, as these
are of primary interest and the only ones used for the estimation of
calibration. However, any existing descending profiles have been tested
and inspected the same way as the ascending profiles, and for cycles where
calibration has been decided, the same correction is done on descending as
on ascending profile.

Technical info on the float is given in Table~\ref{tab:techinfo} and an
overview of the float trajectory and T\&S data is shown in
Figure~\ref{fig:float-info}.
%
\begin{table}[hp]%\footnotesize
\caption{Technical information about float~\WMOnum.}
\label{tab:techinfo}
        \centering
\begin{tabular}{|r|m{12cm}|}
        \hline 
        {WMO float-number} &  {\WMOnum} \\ \hline
        {DAC} & {\input{DAC}} \\ \hline 
        {Float SNR} & {\input{float-serial-no}} \\ \hline  % from file
        {Platform type} & {\input{platform-type}} \\ \hline  % from file
        {Transmission system} & {\input{transmission-system}} \\ \hline 
        {CTD Sensor model} & {\input{ctdmodel}} \\ \hline 
        {CTD Sensor SNR} & {\input{ctdsensorsnr}} \\ \hline 
        {Other sensors} & {\input{othersensors}} \\ \hline 
        {Other sensor models} & {\input{othersensormodels}} \\ \hline 
        {Other sensors SNR} & {\input{othersensorsnr}} \\ \hline 
        {Deployment} & {\input{depdate}} \\ \hline % from file, first profile 
        {Dep. Lat} & {\input{deplat}} \\ \hline  % from file, first profile 
        {Dep. Lon} & {\input{deplon}} \\ \hline  % from file, first profile 
        {Park Depth} & {\input{park-depth}} \\ \hline 
        {Profile depth} & {\input{profile-depth}} \\ \hline 
        {Cycle time} & {\input{cycle-time}} \\ \hline 
        {Ship} & {\input{ship}} \\ \hline
        {PI} & {\input{PI}} \\ \hline 
        {Float Status} & {\input{float-status}} \\ \hline 
        {Age} & {\input{age}} \\ \hline 
        {Last Cycle} & {\input{last-cycle}} \\ \hline 
        {Missing Cycles} & {\input{missing-cycles}} \\ \hline 
        {Grey list} & {\input{grey-list}} \\ \hline
\end{tabular}
\end{table}
        
\begin{figure}[hp]
  %\centering    
  \centerline{\includegraphics[width=1.2\textwidth]{\floatsource\WMOnum}}
  \caption{Float \WMOnum. Map shows the locations of float profiles (numbers in
    black show the first, every 10th, and the last cycle number in the data
    set; numbers in blue with corresponding squares/lines show
    WMO-squares). The grey contours in map indicate bathymetry. The
    following plots are TS-diagram, temperature profiles, salinity
    profiles, and density profiles for all profiles. The profiles shown
    have undergone pre-OWC analysis DMQC (as described in
    Section~\protect\ref{DMQCpreOWC}).  Colour shading in all panels
    indicate cycle number (see colourbar under map).
    Note that for floats with ASD, uncorrectable salinity profiles are not
    shown here.}
  \label{fig:float-info}
\end{figure} 
 
        
%%%%%%%%%%%%%%%%%%%%%%%%%%%%%%%%%%%%%%%%%%%%%%%%%%%%%%%%%%%%%%%%%%%%%%%%%%%%%%%%%%%%%%%%%%
\newpage
\section{Quality Check of Argo Float Data}\label{DMQCpreOWC}
%
The DMQC prior to OWC is performed in four phases, after each the found
erroneous data are removed before the following phase:
\begin{enumerate}
\item Delayed-mode procedures for coordinates
\item Sea Surface Pressure Adjustment (for APEX floats only)
\item Automated tests prior to visual control
\item Visual control of all profiles and flags
\item Overall comparison with the reference data
\end{enumerate}
The tests are described in the following subsections and all results are
shown in Table~\ref{tab:rtqc}.
%
\input{descending/descending_profiles} % Generated by PREPARE_FLOATS.
%
\input{comments}
% 
Note that the flagging summarised and shown in this section, does not
include any flags resulting from OWC analysis (Section~\ref{sec:salcompare}). See
the discussion and conclusions (Section~\ref{sec:discussion}) about any
extra flagging based on the OWC analysis.
%
\begin{table}[!ht]
  \caption{Overview and results for Float~\WMOnum\ in terms of number of
    flags for each variable, from both RTQC and DMQC. 
    Flags based on OWC findings are not shown here.}
  \label{tab:rtqc}
  \centering
  \input{flagtable}
\end{table}     


\subsection{Delayed-mode procedures for coordinates}\label{sec:DM-coordinates}
First the coordinates JULD, LATITUDE, LONGITUDE were checked as prescribed
in Section~3.2 of Wong et al.~(2021).
%
Chronology of JULD was tested by a simple automated test and any missing
or erroneous values replaced by linear interpolation.
\input{JULDtest}

Position outliers were checked for visually in a map such as in
Figure~\ref{fig:float-info} but based on original data, and also in a
time-series reprsentation using the same manual checking tool as described
in Section~\ref{sec:visualDMQC}. Any missing positions have been replaced by 2D
linear interpolation when possible.
\input{POStest}


\newpage
\subsection{Sea Surface Pressure Adjustment}
\input{pressure-adjustment} % Generated by PREPARE_FLOATS.


\newpage
\subsection{Automated tests prior to visual control}\label{sec:automatedDMQC}
In addition to RTQC, the following automated tests are necessary before 
OWC as double checks using same critera as RTQC (see Wong et al., 2021)
and in some aspects stricter based on experience in the region:
\begin{itemize}
\item Pressure increasing test / monotonically increasing pressure test in
  waters deeper than \input{MAP_P_EXCLUDE}m.
\item Double-pointed spike tests on PSAL and TEMP (see Section~\ref{sec:double-pointed-spike-test}).
\item Spike tests on PSAL and TEMP (with 0.02~PSU criterion for pressures
  greater than or equal to 500~dbar, and 0.005~PSU deeper than 1000~dbar;
  and the addition of temperature criterion 1$^\circ$C deeper than 1000~dbar).
\item Gradient test on TEMP and PSAL (declared obsolete from RTQC in 2019,
  but implemented here nevertheless).
\item Density inversion test.
\end{itemize}

\subsubsection{The double-pointed spike test}\label{sec:double-pointed-spike-test}
The double-pointed spike tests the deviation of subsequent pairs of values
instead of single points, from the neighbouring values in a profile, as
this is not an uncommon form of spikes. The test values are formed
according to 
\begin{equation}
  \label{eq:1}
  TV = \left|\frac{V_{i} + V_{i+1}}{2} – \frac{V_{i-1} + V_{i+2}}{2}\right| – \left|\frac{V_{i+1} – V_{i}}{2}\right|  – \left|\frac{V_{i+2} – V_{i-1}}{2}\right| ,
\end{equation}
where $TV$ is the test value and $V_i$ are the subsequent points in a
profile.
%
The first term is the difference between the average of two values forming
a double-spike and the average of their two neighbouring values.
Subtracted are the half width between the respective averaged values, so
that the actual test value is the difference between the nearest two points
of the compared pairs.

As the test values are assigned to the $i$-th
point in the profile (i.e., the upper point of the pair forming the
spike), any testvalues exceeding the criteria results in flagging of both
$V_i$ and $V_{i+1}$.
%
The double-pointed spike test uses the same criteria as the single-point
spike test (see bullet list above).
It is applied before the single-point spike test, since the latter may remove
one of the values in a double-point spike and render the double-pointed spike
test useless.


%\newpage
\subsection{Visual control of all profiles and flags}\label{sec:visualDMQC}
Firstly, an overall check on pressure consistency is done looking at the
lower panel of Figure~\ref{fig:pres}.

Then, as prescribed in Section~3.3--3.5 of Wong et al.~(2021), PRES, TEMP,
and PSAL are compared to temporally close profiles from same float, as
well as surrounding reference data. This is done by inspecting individual
cycles in detail using the tool \emph{check\_profiles} from the toolbox
\href{url}{https://github.com/evenrev1/evenmat.git}.
%
In this tool, profiles for all three parameters as well as their derived
density are plotted on backdrops of profiles from the three preceeding and
three succeeding cycles and localised historical reference data.  Both
NRTQC flags and DMQC flags (Section~\ref{sec:automatedDMQC}) are marked
and can be changed, or new flags can be added, interactively.
%
\input{RTQCcheck} % This part is automated in PREPARE_FLOATS. 


\newpage
\subsection{Overall comparison with the reference data}\label{sec:overall}
In addition to the visual control of individual profiles
(Section~\ref{sec:visualDMQC}), the overall state of the dataset can be
viewed in Figures~\ref{fig:float-info} and \ref{fig:hovmoller}, as well as
all profiles in relation to reference data in the vicinity of the float in
Figure~\ref{fig:refcomp}.

Trends in salinity in the same reference data are also compared to float
data (Figure~\ref{fig:trendcheck}) in order to aid interpretation of any
calibration results in Section~\ref{sec:salcompare}.

\begin{figure}[H] 
  \input{refcomp}  % This part is automated in PREPARE_FLOATS.
\end{figure}
%
\begin{figure}[H] 
  \input{trendcheck}  % This part is automated in PREPARE_FLOATS. 
\end{figure}

 

\newpage
\subsection{Satellite Altimeter Report}
\input{altimetry_sec} % Full subsection with figure or just a sentence
                     % about missing altimetry report.

      
\newpage
\subsection{Time Series of Argo Float Temperature and Salinity}
%
Figure~\ref{fig:hovmoller} shows Hov-M{\"o}ller plots of temperature and
salinty, respectively, disregarding flags but with flagged data marked.
%
Note that for floats with previously assigned uncorrectable profiles, full
profile '4' flags will precede full profile '3' flags, in which case the
latter are the new profiles subject to the latest DMQC session.
%
\begin{figure}[H]
  \centering
  \begin{tabular}{lr}
  a) & \\
  &\includegraphics[width=\textwidth,natwidth=1500,natheight=740]{\floatsource\WMOnum_HMtemp.png}\\
  b)& \\
  &\includegraphics[width=\textwidth,natwidth=1500,natheight=740]{\floatsource\WMOnum_HMsal.png}
  \end{tabular}
  \caption{Float \WMOnum. Time series of Argo float potential temperature
    (a; \textdegree C) and salinity (b; PSS-78). Any white squares and
  circles indicate data that has been flagged '4' by RTQC and DMQC, respectively. 
  A diamond indicates the rare occurence of DMQC reversal of an RTQC flag to '1'.
  White points indicate RTQC flags of '2' or '3'. 
  Flags based on (the current) OWC findings are not shown here (see Section~\ref{sec:discussion} about any extra flags).}
  %   Pressure flags are marked with the same symbols, in grey.
\label{fig:hovmoller}
\end{figure}



\newpage
\section{Correction of Salinity Data}\label{sec:salcorrect}
%
% Deep ARVOR FLOATS PRESSURE DEPENDENT CONDUCTIVITY BIAS 
\input{prescondbias} % Full subsection or empty file

\subsection{Comparison between Argo Float and CTD and Argo reference data}\label{sec:salcompare}
%
In Appendix~B profile plots of reference data from the WMO squares
(\input{wmosquares}) traversed by the float, can be found. These data can
be used in several ways.
%
Important uses are the guidance provided for visual control of Argo
profiles (Section~\ref{sec:visualDMQC}) and the overall comparison with
all reference data in the region covered by the float trajectory
(Section~\ref{sec:overall}).
%
A more advanced use is by the OWC-toolbox, which uses reference data in
order to investigate potential salinity drift and calculate calibration
offsets. The OWC method is essential for DMQC of Argo floats, but depends
on invariable deep waters.
% 
Another possibility is to directly compare float profiles with reference
profiles nearby and near in time, on a one-on-one basis. This can be a
solution for shallow regions (e.g., the Barents Sea and Baltic Sea).
%
The following subsections will reflect which methods have been applied for
Float \WMOnum.

\input{owc} % From RUN_OW_CALIBRATION or empty made in PREPARE_FLOATS or PAIR_FLOATS_WITH_REFERENCEDATA.

\input{coincidence} % From PAIR_FLOATS_WITH_REFERENCEDATA or empty made in PREPARE_FLOATS or RUN_OW_CALIBRATION.



%%%%%%%%%%%%%%%%%%%%%%%%%%%%%%%%%%%%%%%%%%%%%%%%%%%%%%%%%%%%%%%%%%%%%%%%%%%%%%%%%%%%%%%%


\FloatBarrier
\newpage
\section{Discussion and conclusions}\label{sec:discussion}
% THIS FILE IS NECESSARY FOR YOUR DMQCreport_float.tex TO COMPILE.
% THIS IS WHERE YOU WRITE YOUR DISCUSSION/SUMMARY (WE HAVE DECIDED THAT THESE TWO CAN BE THE SAME, FOR SIMPLICITY).
% BELOW ARE SOME EXAMPLE PHRASES THAT YOU CAN COMMENT, UNCOMMENT, EDIT, OR REPLACE AS YOU NEED AND WISH FOR THIS PARTICULAR FLOAT.
%
Float number \WMOnum\ was deployed on \input{depdate}
% centrally in the Norwegian Basin (NB) and has only moved slightly southwestwards, but is still in the deep parts of the basin.
% Hydrography is relatively stable below 1000~m in the profiles.

% Visual verification of in total 232 RTQC flags in 7 profiles was done, of which two had large groups of deviating salinity readings. 
% The DMQC tests resulted in 2 further flags, all minor individual spikes.
% No RTQC flags where present, nor was any new flags necessary in DMQC. 
% The initial comparison between Argo float data and reference data, shows that temperature and salinity data are within normal values, somewhat on the warm and fresh side.
% The comparison with satellite altimeter data shows the same trends, but the float derived DHA has very little of the variability seen in altimetry data.
% The comparison with satellite altimeter data suggested no notable discrepancies.
% The sea surface pressure data are not displaying values below 0~dbar and there are no indications of negative pressure drift.

% The OWC analysis showed no noteworthy indications of drift or offset. Hence, no correction is done.
% The OWC analysis showed a strong negative salinity offset, and float salinity has been corrected with a $\Delta S \approx +0.01$.
% The offset is also evident in long term variability seen in the reference data (Figure~\protect\ref{fig:trendcheck}) and more recent CTD profiles near deployment (Figure~\protect\ref{fig:supplementary}).

\input{monitoring} % DO NOT TOUCH!

%%% Local Variables: 
%%% mode: plain-tex
%%% TeX-master: "DMQCreport_float"
%%% End: 



%%%%%%%%%%%%%%%%%%%%%%%%%%%%%%%%%%%%%%%%%%%%%%%%%%%%%%%%%%%%%%%%%%%%%%%%%%%%%%%%%%%%%%%%



%\newpage
\section*{Acknowledgments}
This report is based on the template given in the DM-REPORT-TEMPLATE
Matlab/LaTeX toolbox provided at\\
\href{url}{https://github.com/euroargodev/dm-report-template.git} and
adapted to own needs (this version is provided in\\
\href{url}{https://github.com/imab4bsh/DMQC-fun.git}).  
%
Calibration of conductivity sensor drift was done using the Matlab OWC
toolbox provided at \href{url}{https://github.com/ArgoDMQC/matlab\_owc}.
%
The map in Figure~\ref{fig:float-info} was made using the M\_MAP toolbox
(Pawlowicz, 2020; \url{http://www.eoas.ubc.ca/~rich/map.html}).
%
The visual DMQC tool CHECK\_PROFILES and other supporting functions can be
found in the author's own distribution at
\href{url}{https://github.com/evenrev1/evenmat.git}.



\section*{References}
\begin{list}{}{}
\item Cabanes, C., Thierry, V., \& Lagadec, C. (2016). Improvement of bias
  detection in Argo float conductivity sensors and its application in the
  North Atlantic. Deep-Sea Research Part I: Oceanographic Research Papers,
  114, 128–136. \href{url}{https://doi.org/10.1016/j.dsr.2016.05.007}.
\item Johnson, G. C., Toole, J. M., \& Larson, N. G. (2007). Sensor
  corrections for Sea-Bird SBE-41CP and SBE-41 CTDs. Journal of
  Atmospheric and Oceanic Technology, 24(6), 1117–1130.
  \href{url}{https://doi.org/10.1175/JTECH2016.1}.
\item Pawlowicz, R., 2020. "M\_Map: A mapping package for MATLAB", version
  1.4m, [Computer software], available online at
  \url{http://www.eoas.ubc.ca/~rich/map.html}.
\item Annie Wong, Robert Keeley, Thierry Carval, and the Argo Data Management Team (2023).
  Argo Quality Control Manual for CTD and Trajectory Data. 
  \href{url}{http://dx.doi.org/10.13155/33951}.
\end{list}

%%\end{flushleft}

\newpage
\FloatBarrier
\section{Appendix A: File information and notes}\label{sec:appendix-notes}
%
\subsection*{Operator's notes}
The following notes have been made about this float:
\verbatiminput{notes}

\input{scientific_calib_tabular} % From WRITE_D or empty made in PREPARE_FLOATS.

\FloatBarrier
\section{Appendix B: Reference data}\label{sec:appendix-refdata}
Here follows overview plots of the reference data in the WMO-squares
traversed by Float~\WMOnum.
\input{appendix}

\FloatBarrier
\section{Appendix C: Supplementary information}\label{sec:supplementary}

\input{shipctd} % From OPERATOR_CPCOR_NEW or empty made in PREPARE_FLOATS.

\input{coincidence_appendix} % From PAIR_FLOATS_WITH_REFERENCEDATA or empty made in PREPARE_FLOATS.

\end{document}
 
