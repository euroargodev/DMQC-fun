\documentclass{article}
\usepackage{graphicx, wrapfig, subcaption, setspace,booktabs,verbatim,hyperref, textcomp, siunitx}
\usepackage{float,placeins}
\usepackage{multirow}
\usepackage{epstopdf}
\usepackage[margin=0.7in]{geometry}

%%%%% Set all your parameters and paths in INIT_DMQC.M (not here) %%%%%%%%%%%%%%%%%%%%%%%
%\newcommand{\WMOnum}{6903550} %% FILL WITH WMONUM
\input{wmonumdef} % Defines \WMOnum.
% Full paths from root are likely necessary here:
\input{floatsourcedef} % Defines \floatsource.
%\newcommand{\floatsource}{/Users/a21627/Arkiv/data/matlab_owc/float_source/project_NorARGO/} 
\input{floatplotsdef} % Defines \floatplots.
%\newcommand{\floatplots}{/Users/a21627/Arkiv/data/matlab_owc/float_plots/project_NorARGO/} 
\input{downloaddirdef} % Defines \downloaddir.
%\newcommand{\downloaddir}{/Users/a21627/Downloads/ARGO/\WMOnum /} 
%%%%%%%%%%%%%%%%%%%%%%%%%%%%%%%%%%%%%%%%%%%%%%%%%%%%%%%%%%%%%%%%
 
\begin{document}

\title{Delayed Mode Quality Control of Argo float~\WMOnum}%\hspace{0.5mm}}
\author{\input{dmqc_operator_name}\\ %Jan Even {\O}ie Nilsen\\   %% Name of DMQC operator
ORCID: \input{dmqc_operator_orcid}\\[5mm] %0000-0003-2516-6106\\[1cm]
\input{dmqc_operator_institution}\\ %Institute of Marine Research (IMR)\\
\input{dmqc_operator_address} } %Strandgaten 196, Bergen, Norway} %% Name of Organisation and address

\date{\today} %% Date of DMQC if you produce report on same day as DMQC.
%\date{yyyy.mm.dd} %% Insert actual date of DMQC if you produce report later.
\maketitle

%\vspace{2cm}
\begin{center}
  \textbf{DMQC summary}
\end{center} 
% THIS FILE IS NECESSARY FOR YOUR DMQCreport_float.tex TO COMPILE.
% THIS IS WHERE YOU WRITE YOUR DISCUSSION/SUMMARY (WE HAVE DECIDED THAT THESE TWO CAN BE THE SAME, FOR SIMPLICITY).
% BELOW ARE SOME EXAMPLE PHRASES THAT YOU CAN COMMENT, UNCOMMENT, EDIT, OR REPLACE AS YOU NEED AND WISH FOR THIS PARTICULAR FLOAT.
%
Float number \WMOnum\ was deployed on \input{depdate}
% centrally in the Norwegian Basin (NB) and has only moved slightly southwestwards, but is still in the deep parts of the basin.
% Hydrography is relatively stable below 1000~m in the profiles.

% Visual verification of in total 232 RTQC flags in 7 profiles was done, of which two had large groups of deviating salinity readings. 
% The DMQC tests resulted in 2 further flags, all minor individual spikes.
% No RTQC flags where present, nor was any new flags necessary in DMQC. 
% The initial comparison between Argo float data and reference data, shows that temperature and salinity data are within normal values, somewhat on the warm and fresh side.
% The comparison with satellite altimeter data shows the same trends, but the float derived DHA has very little of the variability seen in altimetry data.
% The comparison with satellite altimeter data suggested no notable discrepancies.
% The sea surface pressure data are not displaying values below 0~dbar and there are no indications of negative pressure drift.

% The OWC analysis showed no noteworthy indications of drift or offset. Hence, no correction is done.
% The OWC analysis showed a strong negative salinity offset, and float salinity has been corrected with a $\Delta S \approx +0.01$.
% The offset is also evident in long term variability seen in the reference data (Figure~\protect\ref{fig:trendcheck}) and more recent CTD profiles near deployment (Figure~\protect\ref{fig:supplementary}).

\input{monitoring} % DO NOT TOUCH!

%%% Local Variables: 
%%% mode: plain-tex
%%% TeX-master: "DMQCreport_float"
%%% End: 

%\input{summary}

%\begin{flushleft}% text alignment to left
% In this section, write a brief description of all decisions made in DMQC
% analysis. You may include information about sea surface pressure
% corrections with applied QC flags and errors (if applicable), cell thermal
% mass corrections (if applicable), a decision made on salinity data
% including QC flags and corrections applied to salinity data (if needed).
% \vspace{0.5cm}

% For Example: "The sea surface pressure in Apex float was adjusted in
% d-mode. For cycles 1-155, the QC=1 and error 2.4 dbar was assigned to
% pressure data. Cell thermal mass correction was applied.  For cycles
% 1-155, the salty offset was detected. Correction of -0.0125 offset was
% applied, QC=1, error=0.005."

\tableofcontents

\newpage
\section{Introduction}
 
%Briefly write any necessary information about the float parameters,
%location, any steps performed before the DMQC analysis, version and type
%of the reference data, DMQC software and if there any CTD data from
%deployment was used as a reference data and any additional information
%about the deployment issues.     
%\vspace{0.5cm}

This report concerns the delayed mode analysis performed for Argo-float
number~\WMOnum. For more information about this float use, for instance,
the following link:
\begin{center}
\href{url}{http://www.ifremer.fr/argoMonitoring/float/\WMOnum}.  
\end{center}
Before the analysis, real-time QC flags were visually inspected and
modified if necessary.  In addition, a few stricter tests necessary before
the salinity calibration were applied (and flags modified if necessary).
%
Then, the satellite altimeter comparison plot between the sea surface
height and dynamic height anomaly, constructed for this float by Ifremer,
was analysed.
%
Part of this analysis are plots of temperature and salinity time
series, and surface pressure. % \textbf{(if available)}.
%and plots of temperature, salinity and density plotted against the nearby
%historical CTD and Argo profiles.
                                %
%The additional plots supporting the DMQC analysis are included in the ...

% \textbf{Float~\WMOnum\ is the Apex float, where the pressure sensor is not
%   auto-corrected to zero while at the sea surface, hence the pressure data
%   was corrected during processing in delayed mode.  The procedures of
%   correction sea surface pressure are described in Argo Quality Control
%   Manual for CTD and Trajectory Data (Wong et al., 2020).}

% \textbf{The cell thermal mass corrections were applied. For SBE-41 CTDs
%   the estimated alpha was 0.0267 and tau was 18.6 s, for the estimated
%   alpha was 0.141 and tau was 6.68 s (Johnson et al., 2006).}

The salinity calibration has been performed using the configuration and objective
mapping parameters included in Section~\ref{Configuration1}.
                                %and ~\ref{Configuration2}.
The Argo float data were compared to nearby CTD and Argo profiles from the
following reference databases: \texttt{\verbatiminput{refdir}} Reference
data are distributed by Ifremer.  A simple visual check on the reference
data is done prior to analysis (see Appendix~B).

The OWC toolbox version~3.0.0 % Version number needs to be manually updated here
(\href{url}{https://github.com/ArgoDMQC/matlab\_owc}) was run to estimate
a salinity offset and a salinity drift (Cabanes et al., 2016).

Note that only ascending profiles are included in this DMQC.

Technical info on the float is given in Table~\ref{tab:techinfo} and an
overview of the float trajectory and T\&S data is shown in
Figure~\ref{fig:float-info}.
%
\begin{table}[!ht]%\footnotesize
\caption{Technical information about float~\WMOnum.}
\label{tab:techinfo}
        \centering
\begin{tabular}{|r|m{10cm}|}
        \hline 
        {WMO float-number} &  {\WMOnum} \\ \hline
        {DAC} & {\input{DAC}} \\ \hline 
        {Float SNR} & {\input{float-serial-no}} \\ \hline  % from file
        {Platform type} & {\input{platform-type}} \\ \hline  % from file
        {Transmission system} & {\input{transmission-system}} \\ \hline 
        {CTD Sensor model} & {\input{ctdmodel}} \\ \hline 
        {CTD Sensor SNR} & {\input{ctdsensorsnr}} \\ \hline 
        {Other sensors} & {\input{othersensors}} \\ \hline 
        {Other sensor models} & {\input{othersensormodels}} \\ \hline 
        {Other sensors SNR} & {\input{othersensorsnr}} \\ \hline 
        {Deployment} & {\input{depdate}} \\ \hline % from file, first profile 
        {Dep. Lat} & {\input{deplat}} \\ \hline  % from file, first profile 
        {Dep. Lon} & {\input{deplon}} \\ \hline  % from file, first profile 
        {Park Depth} & {\input{park-depth}} \\ \hline 
        {Profile depth} & {\input{profile-depth}} \\ \hline 
        {Cycle time} & {\input{cycle-time}} \\ \hline 
        {Ship} & {\input{ship}} \\ \hline
        {PI} & {\input{PI}} \\ \hline 
        {Float Status} & {\input{float-status}} \\ \hline 
        {Age} & {\input{age}} \\ \hline 
        {Last Cycle} & {\input{last-cycle}} \\ \hline 
        {Grey list} & {\input{grey-list}} \\ \hline
\end{tabular}
\end{table}
        
\begin{figure}[H]
  %\centering    
  \centerline{\includegraphics[width=1.2\textwidth]{\floatsource\WMOnum}}
  \caption{Float \WMOnum. Map shows the locations of float profiles (numbers in
    black are every 10th profile number and numbers in blue with
    corresponding squares/lines show WMO-squares). The grey contours in
    map indicate bathymetry. The following plots are TS-diagram,
    temperature profiles, salinity profiles, and density profiles for all
    profiles. The profiles shown have undergone pre-OWC analysis DMQC (as
    described in Section~\protect\ref{DMQCpreOWC}).  Colour shading in all  
    panels indicate profile number (see colorbar under map).}
  \label{fig:float-info}
\end{figure} 
 
        
%%%%%%%%%%%%%%%%%%%%%%%%%%%%%%%%%%%%%%%%%%%%%%%%%%%%%%%%%%%%%%%%%%%%%%%%%%%%%%%%%%%%%%%%%%
\newpage
\section{Quality Check of Argo Float Data}\label{DMQCpreOWC}
%
The DMQC prior to OWC is performed in four phases, after each the found
erroneous data are removed before the following phase:
\begin{enumerate}
\item Delayed-mode procedures for coordinates
\item Correction of pressure dependent conductivity bias (deep ARVOR only)
\item Visual verification of Real-time mode QC flags
\item Selected automated tests necessary for OWC
\item Visual DMQC of the variables
\end{enumerate}
The tests are described in the following subsections and all results are shown in Table~\ref{tab:rtqc}. 
\input{comments}
%
\begin{table}[!ht]
  \caption{Results for Float~\WMOnum\ in terms of number of flags for each variable, from both RTQC and DMQC.}
  \label{tab:rtqc}
  \centering
  \input{flagtable}
\end{table}     


\subsection{Delayed-mode procedures for coordinates}\label{sec:DM-coordinates}
First the coordinates JULD, LATITUDE, LONGITUDE were checked as prescribed
in Section~3.2 of Wong et al.~(2021).
%
Chronology of JULD was tested by a simple automated test and any missing
or erroneous values replaced by linear interpolation.
%
Position outliers were checked for visually in a map such as in
Figure~\ref{fig:float-info} but based on original data, and any positions
replaced by 2D linear interpolation.

\newpage
\subsection{Visual verification of Real-time mode QC flags}
In the case of RTQC flags in a profile, such profiles are compared to
temporally close profiles from same float, as well as surrounding
reference data. Each individual plot is inspected interactively in detail
and flags are removed or new flags assigned when judged necessary (to both
RTQC flagged data as well as any new findings).
%
\input{RTQCcheck} % This part is automated in LOAD_FLOATS. 
 

\subsection{Selected automated tests necessary for OWC}
In addition to RTQC, the following automated tests are necessary before 
OWC as double checks using same critera as RTQC (see Wong et al., 2021)
and in some aspects stricter based on experience in the region:
\begin{itemize}
\item Pressure increasing test / monotonically increasing pressure test in
  waters deeper than \input{MAP_P_EXCLUDE}m.
\item Double-pointed spike tests on PSAL and TEMP (see Section~\ref{sec:double-pointed-spike-test}).
\item Spike tests on PSAL and TEMP (with 0.02~PSU criterion for pressures
  greater than or equal to 500~dbar, and 0.01~PSU deeper than 1000~dbar;
  and the addition of temperature criterion 1$^\circ$C deeper than 1000~dbar).
\item Gradient test on TEMP and PSAL (declared obsolete from RTQC in 2019,
  but implemented here nevertheless).
\item Density inversion test.
\end{itemize}

\subsubsection{The double-pointed spike test}\label{sec:double-pointed-spike-test}
The double-pointed spike tests the deviation of subsequent pairs of values
instead of single points, from the neighbouring values in a profile, as
this is not an uncommon form of spikes. The test values are formed
according to 
\begin{equation}
  \label{eq:1}
  TV = \left|\frac{V_{i} + V_{i+1}}{2} – \frac{V_{i-1} + V_{i+2}}{2}\right|  – \left|\frac{V_{i+2} – V_{i-1}}{2}\right| ,
\end{equation}
where $TV$ is the test value and $V_i$ are the subsequent points in a
profile. As the test values are assigned to the $i$-th point in the
profile (i.e., the upper point of the pair forming the spike), any testvalues
exceeding the criteria results in flagging of both $V_i$ and $V_{i+1}$.
%
The double-pointed spike test uses the same criteria as the single-point spike test.
It is applied before the single-point spike test, since the latter may remove
one of the values in a double-point spike and render the double-pointed spike
test useless.

\newpage
\subsection{Visual DMQC of the variables}
As prescribed in Section~3.3--3.5 of Wong et al.~(2021), PRES, TEMP, and
PSAL were checked visually by comparing to other cycles from the float
(using versions of Figures~\ref{fig:float-info} and \ref{hovmoller} from
this stage), as well as in relation to reference data in the vicinity of
the float (Figure~\ref{fig:refcomp}).  Trends in salinity in the same
reference data are also compared to float data
(Figure~\ref{fig:trendcheck}) in order to aid interpretation of the
calibration results in Section~\ref{results_CTDArgo}.
%
\begin{figure}[H] 
  \input{refcomp}  % This part is automated in PREPARE_FLOATS.
\end{figure}
%
\begin{figure}[H] 
  \input{trendcheck}  % This part is automated in PREPARE_FLOATS. 
\end{figure}

 

\newpage
\subsection{Satellite Altimeter Report}
Figure~\ref{Altim} shows the comparison with altimetry.
%
\begin{figure}[H]
  \centering    
  \includegraphics[width=\textwidth,natwidth=810,natheight=650]{\downloaddir\WMOnum .png}
  \caption{Float \WMOnum. The comparison between the sea level anomaly
    (SLA) from the satellite altimeter and dynamic height anomaly (DHA)
    extracted from the Argo float temperature and salinity. The figure is
    created by the CLS/Coriolis, distributed by Ifremer
    (ftp://ftp.ifremer.fr/ifremer/argo/etc/argo-ast9-item13-AltimeterComparison/figures/).
    If graphics are missing, an altimetry report is not available (yet).}
  \label{Altim}
\end{figure}

      
\newpage
\subsection{Time Series of Argo Float Temperature and Salinity}
%
Figure~\ref{hovmoller} shows Hov-M{\"o}ller plots of temperature and
salinty, respectively, disregarding flags but with flagged data marked.
% after the pre-OWC DMQC (Section~\ref{DMQCpreOWC}).
\begin{figure}[H]
  \centering
  \begin{tabular}{lr}
  a) & \\
  &\includegraphics[width=\textwidth,natwidth=1500,natheight=740]{\floatsource\WMOnum_HMtemp.png}\\
  b)& \\
  &\includegraphics[width=\textwidth,natwidth=1500,natheight=740]{\floatsource\WMOnum_HMsal.png}
  \end{tabular}
  \caption{Float \WMOnum. Time series of Argo float potential temperature
    (a; \textdegree C) and salinity (b; PSS-78). Any white squares and
  circles indicate data that has been flagged '4' by RTQC and DMQC,
  respectively. A point inside a square indicates the rare occurence of
  reversal of an RTQC flag to '1'. Pressure flags are marked with the same
  symbols, in grey.}
\label{hovmoller}
\end{figure}


\newpage
\subsection{Sea Surface Pressure Adjustment}
\input{pressure-adjustment} % Generated by LOAD_FLOATS.

% % NEI, men plot tid mot overflatetrykk, sensortrykk.
% Sea Surface Pressure Calibrations are not done for
% \input{platform-type}-floats (Wong et al., 2021).  Figure~\ref{surf_press}
% shows the surface pressure from the top of each profile, confirming
% variations less than 5 dbar.
% \begin{figure}[H]
%   \centering    
%   \includegraphics[width=\textwidth,natwidth=1500,natheight=750]{\floatsource\WMOnum_PRES0.png}
%   \caption{Float \WMOnum. Sea surface pressure data. Blue circles indicate
%   pressure value in the real-time.}
% %    \includegraphics[width=\textwidth]{Example_float/surf_pres_\WMOnum}
% %    \caption{Float \WMOnum. Sea surface pressure data. The red crosses indicate the raw pressure before float descent, recorded after sending data to GDAC. Blue circle indicate pressure value in the real-time. Green rotated cross shows the pressure correction applied from the previous float cycle. Top plot- data constrained between -2.4 and 2.4 dbar, middle plot- data constrained between -20 and +20 dbar, bottom plot- data with a max range of data.}
%     \label{surf_press}
% \end{figure}



\newpage
\section{Correction of Salinity Data}

% Deep ARVOR FLOATS PRESSURE DEPENDENT CONDUCTIVITY BIAS 
\input{prescondbias} % Full subsection or empty file

%\subsection{Comparison between Argo Float and CTD Climatlogy}
\subsection{Comparison between Argo Float and CTD and Argo Climatology} 
%
The OWC-toolbox uses reference data in order to investigate potential
salinity drift and calculate calibration offsets. Figure~\ref{trajectory}
shows the positions of reference data actually used in mapping. In 
Appendix~B profile plots of reference data from the WMO squares
(\input{wmosquares}) traversed by the float, can be found.

\subsubsection{Configuration}\label{Configuration1}
% The following is a trascript of the \texttt{ow\_config.txt} file of the
% OWC toolbox with the parameters set for the final correction, as well as
% comments added by the operator.
% \verbatiminput{ow_config.txt} 
The following are the mapping configuration parameters set in
\texttt{ow\_config.txt} file of the OWC toolbox with the parameters set
for the final correction:  \verbatiminput{ow_config}
The scaling parameters are typical for use in the Lofoten Basin. % (LBmap, KAMage, KAMphi)

The calseries parameters are set in \texttt{set\_calseries.m} file as
follows: \verbatiminput{cal_par}


\subsubsection{Results}\label{results_CTDArgo}
Figures \ref{trajectory} through \ref{Salinity_OWlevels} show the results
of the comparison and correction of the salinity data.
%
Notes made about this float during the different rounds of DMQC, can be
found in Appendix~A.


\begin{figure}[ph]
  \centering    
  \includegraphics[width=\textwidth,natwidth=450,natheight=330]{\floatplots\WMOnum_1.eps}
  \caption{Float \WMOnum. Location of the float profiles (red line with
  black numbers) and the reference data selected for mapping (blue
  dots).}
  \label{trajectory}
\end{figure}
\begin{figure}[ph]
    \centering    
    \includegraphics[width=\textwidth,natwidth=1400,natheight=820]{\floatplots\WMOnum_3}
    \caption{Float \WMOnum. Evolution of the suggested adjustment with
    time. The top panel plots the potential conductivity multiplicative
    adjustment. The bottom panel plots the equivalent salinity additive
    adjustment. The red line denotes one-to-one profile fit that uses the
    vertically weighted mean of each profile. The red line can be used to
    check for anomalous profiles relative to the optimal fit.} 
    \label{SalWithErrors}
\end{figure}
\begin{figure}[p]
    \centering    
    \includegraphics[width=\textwidth,natwidth=1000,natheight=900]{\floatplots\WMOnum_2}
    \caption{Float \WMOnum. The original float salinity and the
      objectively estimated reference salinity at the 10 float theta
      levels that are used in calibration as errorbars. Lower panel is a
      zoom to the latter.}
    \label{uncalibVsSalinity}
\end{figure}
\begin{figure}[p]
    \centering    
    \includegraphics[width=\textwidth,natwidth=1000,natheight=900]{\floatplots\WMOnum_4}
    \caption{Float \WMOnum. Plots of calibrated float salinity and the
      objectively estimated reference salinity at the 10 float theta
      levels that are used in calibration as errorbars. Lower panel is a
      zoom to the latter.}
    \label{CalibVsSalinity}
\end{figure}
\begin{figure}[p]
    \centering     
    \includegraphics[width=0.8\textwidth,natwidth=500,natheight=700]{\floatplots\WMOnum_5}
    \caption{Float \WMOnum. Salinity anomaly on theta levels.}
    \label{SalAnomOnTheta}
\end{figure}
\begin{figure}[p]
    \centering    
    \includegraphics[width=0.8\textwidth,natwidth=500,natheight=700]{\floatplots\WMOnum_7}
    \caption{Float \WMOnum.  Calibrated salinity anomaly on theta levels.}
    \label{CalibSalAnomOnTheta}
\end{figure}
\begin{figure}[p]
    \centering    
    \includegraphics[width=\textwidth,natwidth=1350,natheight=820]{\floatplots\WMOnum_6}
    \caption{Float \WMOnum. Plots of the evolution of salinity with time
    along with selected theta levels with minimum salinity variance.} 
    \label{SalErrOnTheta}
\end{figure}
\begin{figure}[p]
  \centerline{\includegraphics[width=1.2\textwidth,natwidth=1700,natheight=1000]{\floatplots\WMOnum_8}}
  \caption{Float \WMOnum. Plots include the theta levels chosen for
    calibration: Top left: Salinity variance at theta levels. Top right:
    T/S diagram of all profiles of Argo float. Bottom left: potential
    temperature plotted against pressure. Bottom right: salinity plotted
    against pressure.} 
  \label{Salinity_OWlevels}
\end{figure}

\FloatBarrier
\newpage
\section{Discussion and conclusions}  
% THIS FILE IS NECESSARY FOR YOUR DMQCreport_float.tex TO COMPILE.
% THIS IS WHERE YOU WRITE YOUR DISCUSSION/SUMMARY (WE HAVE DECIDED THAT THESE TWO CAN BE THE SAME, FOR SIMPLICITY).
% BELOW ARE SOME EXAMPLE PHRASES THAT YOU CAN COMMENT, UNCOMMENT, EDIT, OR REPLACE AS YOU NEED AND WISH FOR THIS PARTICULAR FLOAT.
%
Float number \WMOnum\ was deployed on \input{depdate}
% centrally in the Norwegian Basin (NB) and has only moved slightly southwestwards, but is still in the deep parts of the basin.
% Hydrography is relatively stable below 1000~m in the profiles.

% Visual verification of in total 232 RTQC flags in 7 profiles was done, of which two had large groups of deviating salinity readings. 
% The DMQC tests resulted in 2 further flags, all minor individual spikes.
% No RTQC flags where present, nor was any new flags necessary in DMQC. 
% The initial comparison between Argo float data and reference data, shows that temperature and salinity data are within normal values, somewhat on the warm and fresh side.
% The comparison with satellite altimeter data shows the same trends, but the float derived DHA has very little of the variability seen in altimetry data.
% The comparison with satellite altimeter data suggested no notable discrepancies.
% The sea surface pressure data are not displaying values below 0~dbar and there are no indications of negative pressure drift.

% The OWC analysis showed no noteworthy indications of drift or offset. Hence, no correction is done.
% The OWC analysis showed a strong negative salinity offset, and float salinity has been corrected with a $\Delta S \approx +0.01$.
% The offset is also evident in long term variability seen in the reference data (Figure~\protect\ref{fig:trendcheck}) and more recent CTD profiles near deployment (Figure~\protect\ref{fig:supplementary}).

\input{monitoring} % DO NOT TOUCH!

%%% Local Variables: 
%%% mode: plain-tex
%%% TeX-master: "DMQCreport_float"
%%% End: 



%%%%%%%%%%%%%%%%%%%%%%%%%%%%%%%%%%%%%%%%%%%%%%%%%%%%%%%%%%%%%%%%%%%%%%%%%%%%%%%%%%%%%%%%



%\newpage
\section*{Acknowledgments}
This report is based on the template given in the DM-REPORT-TEMPLATE
Matlab/LaTeX toolbox provided at
\href{url}{https://github.com/euroargodev/dm-report-template.git} and
adapted to own needs (this version is provided in
\href{url}{https://github.com/imab4bsh/DMQC-fun.git}).  
%
Calibration of conductivity sensor drift was done using the Matlab OWC
toolbox provided at \href{url}{https://github.com/ArgoDMQC/matlab\_owc}.
%
The map in Figure~\ref{fig:float-info} was made using the M\_MAP toolbox
(Pawlowicz, 2020; \url{http://www.eoas.ubc.ca/~rich/map.html}).
%
Supporting functions can be found in the author's own distribution at
\href{url}{https://github.com/evenrev1/evenmat.git}.



\section*{References}
\begin{list}{}{}
\item Cabanes, C., Thierry, V., \& Lagadec, C. (2016). Improvement of bias
  detection in Argo float conductivity sensors and its application in the
  North Atlantic. Deep-Sea Research Part I: Oceanographic Research Papers,
  114, 128–136. \href{url}{https://doi.org/10.1016/j.dsr.2016.05.007}.
\item Johnson, G. C., Toole, J. M., \& Larson, N. G. (2007). Sensor
  corrections for Sea-Bird SBE-41CP and SBE-41 CTDs. Journal of
  Atmospheric and Oceanic Technology, 24(6), 1117–1130.
  \href{url}{https://doi.org/10.1175/JTECH2016.1}.
\item Pawlowicz, R., 2020. "M\_Map: A mapping package for MATLAB", version
  1.4m, [Computer software], available online at
  \url{http://www.eoas.ubc.ca/~rich/map.html}.
\item Wong,~A., Keeley,~R., Carval,~T., and the Argo Data Management Team
  (2021).  Argo Quality Control Manual for CTD and Trajectory Data.
  \href{url}{http://dx.doi.org/10.13155/33951}.
\end{list}

%%\end{flushleft}

%\end{document}
% ADD FILENAMES MANUALLY BELOW!!

\newpage
\section{Appendix A: File information and notes}\label{sec:appendix-notes}
%
\subsection*{Scientific calibration information}
The scientific calibration information written to the D-files are
summarized in Table~\ref{tab:scientific_calib}.
%
\begin{table}[!h]
  \caption{Information filled in the SCIENTIFIC\_CALIB section for the variables, in the D-files.} \label{tab:scientific_calib} 
  \input{scientific_calib_tabular}
\end{table}

\subsection*{Operator's notes}
The following notes have been made about this float:
\verbatiminput{notes}


\section{Appendix B: Reference data}\label{sec:appendix-refdata}
Here follows overview plots of the reference data in the WMO-squares
traversed by Float~\WMOnum.
\input{appendix}

% % Template for supplementary material. Use this file and refer to figures here in discussion.tex.
% \FloatBarrier
% \section{Appendix C: Supplementary information}\label{sec:supplementary}
% %
% \begin{figure}[ph]
% \centering
% \includegraphics[width=0.8\textwidth,natwidth=600,natheight=500]{6903563_salt.jpg}
% \caption{Comparison of first float profiles and nearby ship CTD-stations near the time of deployment.}
% \label{fig:supplementary}
% \end{figure}

%%% Local Variables: 
%%% mode: plain-tex
%%% TeX-master: "DMQCreport_float"
%%% End: 
 % If anything extra is made, it is done in this
                      % file. Otherwise file will be empty made in PREPARE_FLOATS.


\end{document}
 
