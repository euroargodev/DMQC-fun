% THIS FILE IS NECESSARY FOR YOUR DMQCreport_float.tex TO COMPILE.
% THIS IS WHERE YOU WRITE YOUR DISCUSSION/SUMMARY (WE HAVE DECIDED THAT THESE TWO CAN BE THE SAME, FOR SIMPLICITY).
% BELOW ARE SOME EXAMPLE PHRASES THAT YOU CAN COMMENT, UNCOMMENT, EDIT, OR REPLACE AS YOU NEED AND WISH FOR THIS PARTICULAR FLOAT.
%
Float number \WMOnum\ was deployed on \input{depdate}
% centrally in the Norwegian Basin (NB) and has only moved slightly southwestwards, but is still in the deep parts of the basin.
% Hydrography is relatively stable below 1000~m in the profiles.

% Visual verification of in total 232 RTQC flags in 7 profiles was done, of which two had large groups of deviating salinity readings. 
% The DMQC tests resulted in 2 further flags, all minor individual spikes.
% No RTQC flags where present, nor was any new flags necessary in DMQC. 
% The initial comparison between Argo float data and reference data, shows that temperature and salinity data are within normal values, somewhat on the warm and fresh side.
% The comparison with satellite altimeter data shows the same trends, but the float derived DHA has very little of the variability seen in altimetry data.
% The comparison with satellite altimeter data suggested no notable discrepancies.
% The sea surface pressure data are not displaying values below 0~dbar and there are no indications of negative pressure drift.

% The OWC analysis showed no noteworthy indications of drift or offset. Hence, no correction is done.
% The OWC analysis showed a strong negative salinity offset, and float salinity has been corrected with a $\Delta S \approx +0.01$.
% The offset is also evident in long term variability seen in the reference data (Figure~\protect\ref{fig:trendcheck}) and more recent CTD profiles near deployment (Figure~\protect\ref{fig:supplementary}).

\input{monitoring} % DO NOT TOUCH!

%%% Local Variables: 
%%% mode: plain-tex
%%% TeX-master: "DMQCreport_float"
%%% End: 
